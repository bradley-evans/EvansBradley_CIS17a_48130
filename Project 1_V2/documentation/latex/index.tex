\hypertarget{index_intro_sec}{}\section{Introduction}\label{index_intro_sec}
This game is meant to take place in a single room, with the player acting as interrogator questioning a terrorism suspect. The player will reveal different new \mbox{[}subjects\mbox{]} through interrogation, and based on how the questions are asked and how different coercion techniques are employed, the suspect will either tell the truth, lie, or not answer at all.

The suspect is governed by statistics that are generated randomly before each session, and a general description of the suspect is given based on what each statistic reads.\hypertarget{index_known_issues}{}\section{Known Problems}\label{index_known_issues}
This build does not have complete functionality. For one thing, the only command that works with any real reliability is \char`\"{}ask name,\char`\"{} and there are a few strange input errors related to my implementation of getline. This program is meant more as a proof of concept -- a skeleton upon which to build a more complex and detail-\/rich game.

I attempted to follow conventions for header and external .cpp files, but when I began moving functions out of the header files and into proper .cpp files, the program ceased to compile. Documentation of the worst error was posted here on Stack\+Exchange\+:

\href{http://stackoverflow.com/questions/26569177/compiler-error-enumerated-class-was-not-declared-in-this-scope?noredirect=1#comment41758610_26569177}{\tt http\+://stackoverflow.\+com/questions/26569177/compiler-\/error-\/enumerated-\/class-\/was-\/not-\/declared-\/in-\/this-\/scope?noredirect=1\#comment41758610\+\_\+26569177}

Sometimes, when the command 'quit' is used, the game crashes rather than exits normally.\hypertarget{index_explain}{}\section{How it Works}\label{index_explain}
1 -\/ User enters a command 2 -\/ Command is taken as a cstring and parsed out into two std\+::strings, assuming that the first word is a verb and the second is a noun. 3 -\/ The std\+::strings are compared against maps of nouns and verbs, and then converted into an associated enumerated value. 4 -\/ The enumerated values then drive switch statements which decide what output to give to the user. 5 -\/ Outcomes, statistics, the \char`\"{}truth table,\char`\"{} and other ingame conditions are modified by the choice, and then a new command is requested.\hypertarget{index_requirements}{}\section{Assignment Requirements}\label{index_requirements}
\hypertarget{index_malloc}{}\subsection{Memory allocation}\label{index_malloc}
An admittedly crude example of this can be found in the \hyperlink{class_dice}{Dice} class.\hypertarget{index_structs}{}\subsection{Functions with structures}\label{index_structs}
See \char`\"{}classes.\+h\char`\"{} The diceroller used in this system includes functions with structures. The gameclock is initialized and iterated in this way as well.\hypertarget{index_arraystructs}{}\subsection{Pointers with arrays and arrays of structures, internally as well as externally.}\label{index_arraystructs}
Commands in this game are taken in by character arrays and passed around by reference to other parts of the parsing process.

An \char`\"{}array of structures\char`\"{} was implemented to record \char`\"{}high scores\char`\"{} in \hyperlink{functions_8h_source}{functions.\+h}, the total\+Scores function.

A 2\+D array is implemented in print\+Line(); under cmdtree. 